%
%
% UCSD Doctoral Dissertation Template
% -----------------------------------
% http://ucsd-thesis.googlecode.com
%
%


%% REQUIRED FIELDS -- Replace with the values appropriate to you

% No symbols, formulas, superscripts, or Greek letters are allowed
% in your title.
\title{Statistical Modeling of Sensors and Dials in Metabolic Networks}

\author{Swathi Vishwanath Hoysala}
\degreeyear{2018}

% Master's Degree theses will NOT be formatted properly with this file.
\degreetitle{Master of Science}

\field{Computer Science and Engineering}
%\specialization{Anthropogeny}  % If you have a specialization, add it here

\chair{Professor Bernhard O. Palsson}
% Uncomment the next line iff you have a Co-Chair
% \cochair{Professor Cochair Semimaster}
%
% Or, uncomment the next line iff you have two equal Co-Chairs.
%\cochairs{Professor Chair Masterish}{Professor Chair Masterish}

%  The rest of the committee members  must be alphabetized by last name.
\othermembers{
Professor Nuno Bandeira\\
Professor Debashis Sahoo\\
}
\numberofmembers{3} % |chair| + |cochair| + |othermembers|


%% START THE FRONTMATTER
%
\begin{frontmatter}

%% TITLE PAGES
%
%  This command generates the title, copyright, and signature pages.
%
\makefrontmatter

%% DEDICATION
%
%  You have three choices here:
%    1. Use the ``dedication'' environment.
%       Put in the text you want, and everything will be formated for
%       you. You'll get a perfectly respectable dedication page.
%
%
%    2. Use the ``mydedication'' environment.  If you don't like the
%       formatting of option 1, use this environment and format things
%       however you wish.
%
%    3. If you don't want a dedication, it's not required.
%
%
\begin{dedication}
%  To two, the loneliest number since the number one.
To my Mom, Dad, Grandparents and Sister.\\ I am what I am because of you!
\end{dedication}


% \begin{mydedication} % You are responsible for formatting here.
%   \vspace{1in}
%   \begin{flushleft}
% 	To me.
%   \end{flushleft}
%
%   \vspace{2in}
%   \begin{center}
% 	And you.
%   \end{center}
%
%   \vspace{2in}
%   \begin{flushright}
% 	Which equals us.
%   \end{flushright}
% \end{mydedication}



%% EPIGRAPH
%
%  The same choices that applied to the dedication apply here.
%
%\begin{epigraph} % The style file will position the text for you.
%  \emph{A careful quotation\\
%  conveys brilliance.}\\
%  ---Smarty Pants
%\end{epigraph}

% \begin{myepigraph} % You position the text yourself.
%   \vfil
%   \begin{center}
%     {\bf Think! It ain't illegal yet.}
%
% 	\emph{---George Clinton}
%   \end{center}
% \end{myepigraph}


%% SETUP THE TABLE OF CONTENTS
%
\tableofcontents
\listoffigures  % Comment if you don't have any figures
\listoftables   % Comment if you don't have any tables



%% ACKNOWLEDGEMENTS
%
%  While technically optional, you probably have someone to thank.
%  Also, a paragraph acknowledging all coauthors and publishers (if
%  you have any) is required in the acknowledgements page and as the
%  last paragraph of text at the end of each respective chapter. See
%  the OGS Formatting Manual for more information.
%
\begin{acknowledgements}
 My advisor Dr. Bernhard O. Palsson for taking me under his wing, for seeing the potential in this project and for his valuable feedback.  \\
 My co-advisor and mentor Dr. Zachary King for being my neighbor who I happened to bump into in the hallway and decided to collaborate. For the countless hours he spent explaining the biological concepts, for the brainstorming sessions and most importantly for his research acumen.\\
 My committee members Dr. Nuno Bandeira and Dr. Debashis Sahoo for their valuable and insightful feedback.\\ 
 Researchers at System's Biology Research Group for all their help and support.\\
 The Novo Nordisk foundation for funding my research. \\
 Isaac Shamie for laying the initial groundwork for this project.\\
 My friend Alok for all the caffeine and entertainment he supplied and my roommate Asmitha for feeding me\\
 
\end{acknowledgements}


%% VITA
%
%  A brief vita is required in a doctoral thesis. See the OGS
%  Formatting Manual for more information.
%
%\begin{vitapage}
%\begin{vita}
%  \item[2002] B.~S. in Mathematics \emph{cum laude}, University of Southern North Dakota, %Hoople
%  \item[2002-2007] Graduate Teaching Assistant, University of California, San Diego
%  \item[2007] Ph.~D. in Mathematics, University of California, San Diego
%\end{vita}
%\begin{publications}
%  \item Your Name, ``A Simple Proof Of The Riemann Hypothesis'', \emph{Annals of Math}, %314, 2007.
%  \item Your Name, Euclid, ``There Are Lots Of Prime Numbers'', \emph{Journal of Primes}, %1, 300 B.C.
%\end{publications}
%\end{vitapage}


%% ABSTRACT
%
%  Doctoral dissertation abstracts should not exceed 350 words.
%   The abstract may continue to a second page if necessary.
%
\begin{abstract}
 Knowing how a microbe senses environmental inputs and regulates metabolic changes is important for metabolic engineers trying to direct microbial resources and reactions to specific pathways. Prediction of metabolic changes that result from genetic or environmental perturbations has several important applications, including diagnosing metabolic disorders and discovering novel drug targets. Most of the research in the field of modeling transcriptional regulatory networks (TRNs) and their metabolic effects focuses on integrating metabolic networks with additional data like transcriptional or genomic data. However, these existing methods are limited by the availability of datasets and the huge parameter space associated with TRN models. Thus, there is a need for alternative approaches to modeling regulation of metabolic networks.
 
It was recently established that microbial cells consist of flux sensors which measure the rate at which enzymatic reactions takes place, and then adjust, or dial, certain reactions and pathway fluxs. We hypothesize that these flux \textbf{sensors provide enough information to predict the change in metabolic ``dials''}, i.e flux splits between different pathways. This project aims to prove the above-mentioned hypothesis using statistical modeling of sensors and dials data in the metabolic network simulations. 
  
Using Markov Chain Monte Carlo sampling methods, we sample the flux states of the \textit{Escherichia coli} K-12 MG1655 strain under varying nutrient sources. We sample from 34 conditions  to create a dataset with 340000 datapoints, each representing a unique feasible metabolic flux. We then apply statistical modeling techniques including linear regression, decision trees and ensemble learning methods to predict metabolic dial values using sensor values as input. The results from the statistical modeling techniques show that sensors can effectively predict the dial values without the need for additional data like transcriptional or genomic data.

\end{abstract}


\end{frontmatter}
